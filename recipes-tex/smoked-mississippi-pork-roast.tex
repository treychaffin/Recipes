\documentclass[../recipe-collections/cooking.tex]{subfiles}
\externaldocument[V-]{../information/vendors} 

\begin{document}
\begin{recipe}{\textbf{Smoked Mississippi Pork Roast}}{}{}

  \freeform{}\textit{Smoked Mississippi Pork Roast}

  \ing[~8]{pound}{\href{https://www.dillons.com/p/kroger-pork-shoulder-butt/0020799300000}{Pork Shoulder Butt}}
  \ing[~2]{tablespoons}{\href{https://www.dillons.com/p/kinder-s-butcher-s-all-purpose-seasoning/0075579537545}{All-Purpose Meat Rub}}

  Remove the pork butt from the packaging, wipe dry with some paper towels.
  Cover the pork with the all-purpose rub. Cook the pork on the smoker at 275F
  until the internal temperature has reached 160F.

  \ing[8]{tablespoons}{\href{https://www.dillons.com/p/kroger-unsalted-butter-sticks/0001111089305}{Butter}}
  \ing[1]{packet}{\href{https://www.dillons.com/p/kroger-ranch-dip-mix/0001111070875}{Ranch}}
  \ing[1]{packet}{\href{https://www.dillons.com/p/kroger-pork-flavored-gravy-mix/0001111071993}{Pork Gravy}}
  \ing[8]{}{\href{https://www.dillons.com/p/kroger-pepperoncini-peppers/0001111009087}{Pepperoncini Peppers}}

  Once the pork has reached an internal temperature of 160F, pull the pork
  from the smoker. Prepare some aluminum foil pans and aluminum foil and 
  place the pork in the pan. Empty the gravy and ranch packets across the pork.
  Slice the butter and distribute it evenly across the top of the pork. Put the 
  pepperoncini peppers on the pork as well. Wrap the pork with foil and place
  the pork back into the smoker. Continue cooking the pork until it has reached
  an internal temperature of 200F. After it has reached temperature, pull the pork
  from the smoker and let it rest for at least 30 mins. Shred the pork.

  \freeform{}\hrulefill{}

\end{recipe}

Re-produced from \citetitle{HowToBBQRight_PorkRoast_2022} \autocite{HowToBBQRight_PorkRoast_2022}

\end{document}