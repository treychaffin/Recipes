\documentclass[../recipe-collections/cooking.tex]{subfiles}

\begin{document}
\begin{recipe}{\textbf{Mac \& Cheese}}{4 servings}{20m prep time, 20m cook time}

  \freeform{}\textit{Mac \& Cheese}

  \ing[6-8]{cups}{Cauliflower}
  \ing[2]{slices}{Bacon, thick cut}  

  Separate the Cauliflower into macaroni sized pieces. Dice the bacon. Using a skillet,
  sweat the bacon on low heat, then add the cauliflower and continue cooking until the 
  bacon is cooked. 

  \ing[\fr12]{medium}{Onion, diced}
  \ing[2]{cloves}{Garlic, finely chopped}

  Add the onions and garlic, continue to cook until the onions are transparent.

  \ing[2]{tbsp}{Butter}
  \ing[1]{cup}{Cheddar Cheese, shredded}
  \ing[\fr14]{cup}{Heavy Whipping Cream}
  \ing[]{}{Salt \& Pepper}

  Combine the butter, cheese, and heavy cream in a saucepan. Cook on low heat until everything
  is melted together. Add salt \& pepper to taste.

  \ing[]{}{Pork Rinds, crushed}

  Place the cauliflower mixture into a 8x8 baking dish. Pour the cheese sauce over top of the mix.
  Top with crush pork rinds. Bake for 10 mins at 425\0 and then broil for 3-5 to crisp topping.

  \bigskip
  \centering
  \begin{tabular}{|lr|}
    \hline
                                        &                                       \\
    \multicolumn{2}{|l|}{\huge{\textbf{\textrm{Nutrition Facts}}}}              \\
    \multicolumn{2}{|l|}{\textrm{Mac \& Cheese}}                                \\ 
                                        &                                       \\
    \multicolumn{2}{|l|}{\footnotesize{\textbf{\textrm{Amount per Serving}}}}   \\ \hline
    \textbf{\textrm{Calories (kcal)}}   & \textrm{315}                          \\ \hline
    \textbf{\textrm{Fat (g)}}           & \textrm{25.5}                         \\ \hline
    \textbf{\textrm{Carbohydrates (g)}} & \textrm{7.5}                          \\ \hline
    \hspace{2mm} \textrm{Fiber (g)}     & \textrm{2}                            \\ \hline
    \textbf{\textrm{Protein (g)}}       & \textrm{16.5}                         \\ \hline
  \end{tabular}

  \freeform{}\hrulefill{}

\end{recipe}

Re-produced from \citetitle{gaedke_2021} \autocite{gaedke_2021}

\end{document}