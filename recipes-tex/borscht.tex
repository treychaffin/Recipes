\documentclass[../recipe-collections/cooking.tex]{subfiles}

\begin{document}
\begin{recipe}{\textbf{Borscht}}{4 servings}{}

  \freeform{}\textit{Brothy and brimming with beets, parsnips, turnip, 
  celery root, and slices of kielbasa, this earthy beet soup gets a 
  finishing touch of sour cream and fresh dill. Serve it in big bowls 
  with plenty of crusty bread for an appetizing cold-weather dinner}

  \ing[2]{tbsp}{cooking oil}
  \ing[1]{}{onion, chopped}
  \ing[3/4 lb]{}{celery root}
  \ing[1]{}{turnip, peeled and cut to 1/2 inch chunks}
  \ing[1 3/4]{tsp}{salt}

  In a large saucepan, heat the oil over moderately low heat. Add the onion 
  and cook, stirring occasionally, until translucent, about 5 minutes. 
  Add the parsnips, celery root, turnip, and 1 teaspoon of the salt. 
  Cover and cook until the vegetables start to soften, about 5 minutes.

  \ing[2]{cups}{drained diced canned beets (one 15 ounce can)}
  \ing[1 1/2]{cups}{drained diced tomatoes (one 15 ounce can)}
  \ing[3 1/2]{cups}{beef stock}
  \ing[3]{cups}{water}
  \ing[1/4]{tsp}{ground black pepper}
  \ing[1/2]{lb}{kielbasa, halved lengthwise and sliced crosswise}

  Add the beets, tomatoes, broth, water, the remaining 3/4 teaspoon salt, 
  and the pepper. Bring to a boil. Add the kielbasa. Reduce the heat and 
  simmer, uncovered, until the vegetables are tender, about 15 minutes. 
  Stir in the 1/3 cup dill. Serve topped with the sour cream and the 
  remaining 3 tablespoons dill.

  \ing[8]{tbsp}{fresh dill, chopped}
  \ing[1/4]{cup}{sour cream}

  Serve topped with the sour cream and the remaining 3 tablespoons dill.

  % \bigskip
  % \centering
  % \begin{tabular}{|lr|}
  %   \hline
  %                                       &                                       \\
  %   \multicolumn{2}{|l|}{\huge{\textbf{\textrm{Nutrition Facts}}}}              \\
  %   \multicolumn{2}{|l|}{\textrm{Borscht}}                               \\ 
  %                                       &                                       \\
  %   \multicolumn{2}{|l|}{\footnotesize{\textbf{\textrm{Amount per Serving}}}}   \\ \hline
  %   \textbf{\textrm{Calories (kcal)}}   & \textrm{122}                          \\ \hline
  %   \textbf{\textrm{Fat (g)}}           & \textrm{14}                           \\ \hline
  %   \textbf{\textrm{Carbohydrates (g)}} & \textrm{7}                            \\ \hline
  %   \textbf{\textrm{Protein (g)}}       & \textrm{25}                            \\ \hline
  % \end{tabular}

  \freeform{}\hrulefill{}

\end{recipe}

Re-produced from \citetitle{FoodAndWine_ChunkyBorscht_2019} 
\autocite{FoodAndWine_ChunkyBorscht_2019}

\end{document}