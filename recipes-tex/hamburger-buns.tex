\documentclass[../recipe-collections/cooking.tex]{subfiles}

\begin{document}
\begin{recipe}{\textbf{Hamburger Buns}}{5 buns}{15m prep time, 12m cook time}

  \freeform{}\textit{Cheesy Almond Flour Hamburger Buns}

  \ing[1 \fr12]{cups}{Mozzarella Cheese, shredded}
  \ing[2]{ounces}{Cream Cheese}
  
  Place the mozzarella and cream cheese in the microwave for 1 minute. Stir the cheeses. 
  Continue to microwave in 30 second intervals and stiring the mixture until well mixed.

  \ing[1]{large}{Egg}

  Mix the egg into the cheese mixture until smooth.

  \ing[1 \fr12]{cups}{Almond Flour}
  \ing[2]{tbsp}{Baking Powder}
  
  Mix the almond flour and baking powder first, then slowly mix the dry ingredients
  into the cheese mixture until a dough forms. Divide the dough into 5 equal peices 
  and form the peices to a bun shape. Line a baking sheet with parchment paper and place
  the peices on the sheet.

  \ing[]{}{Sesame Seeds}
  \ing[1]{large}{Egg}

  Whip the egg together until smooth. Use a brush to apply the egg mix to the buns.
  Sprinkle sesame seeds onto buns. 

  Heat the oven to 400\0F.  Place a metal pan with ice cubes in it at the 
  bottom of the oven (helps dough rise). Bake for 12 minutes or until the 
  outside of the buns has browned.

  \bigskip
  \centering
  \begin{tabular}{|lr|}
    \hline
                                        &                                       \\
    \multicolumn{2}{|l|}{\huge{\textbf{\textrm{Nutrition Facts}}}}              \\
    \multicolumn{2}{|l|}{\textrm{Hamburger Buns}}                               \\ 
                                        &                                       \\
    \multicolumn{2}{|l|}{\footnotesize{\textbf{\textrm{Amount per Serving}}}}   \\ \hline
    \textbf{\textrm{Calories (kcal)}}   & \textrm{294}                          \\ \hline
    \textbf{\textrm{Fat (g)}}           & \textrm{25}                           \\ \hline
    \textbf{\textrm{Carbohydrates (g)}} & \textrm{7}                            \\ \hline
    \hspace{2mm} \textrm{Fiber (g)}     & \textrm{3}                            \\ \hline
    \textbf{\textrm{Protein (g)}}       & \textrm{14}                            \\ \hline
  \end{tabular}

  \freeform{}\hrulefill{}

\end{recipe}

Re-produced from \citetitle{hardesty_2021} \autocite{hardesty_2021}

\end{document}