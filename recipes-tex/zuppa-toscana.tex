\documentclass[../recipe-collections/cooking.tex]{subfiles}

\begin{document}

\begin{recipe}{\textbf{Zuppa Toscana}}{6 servings}{20m prep time, 20min cook time}
  
  \freeform{}\textit{Zuppa Toscana Soup}

  \ing[1]{pound}{Italian Sausage, Hot or Mild}
  
  In a soup pot on the stove, crumble and brown sausage over medium-high heat.

  \ing[1]{}{Onion, Diced}
  \ing[3]{}{Garlic Cloves, Minced}
  
  Add onion and garlic and cook until translucent in color. 
  
  \ing[\fr12]{tsp}{Red Pepper Flakes}
  \ing[\fr14]{tsp}{Black Pepper}
  \ing[\fr12]{tsp}{Salt}
  
  Season with salt, pepper, and red pepper flakes. 
  
  \ing[1]{head}{Cauliflower, Cut Into Florets}
  \ing[16]{ounces}{Chicken Broth}
  \ing[1]{quart}{Water}
  \ing[1]{tsp}{Chicken Bouillon}
  
  Reduce heat to medium and add cauliflower florets, broth, and water. 
  Stir and add bouillon, if desired. Cook on medium heat until 
  cauliflower is tender, about 15--20 minutes. 
  
  \ing[3]{cups}{Kale, Chopped}
  \ing[1]{cup}{Heavy Cream}

  Reduce heat to low and sprinkle in chopped kale. Pour in cream and stir well. Serve hot.
  
  \bigskip
  \centering
  \begin{tabular}{|lr|}
    \hline
                                        &                                       \\
    \multicolumn{2}{|l|}{\huge{\textbf{\textrm{Nutrition Facts}}}}              \\
    \multicolumn{2}{|l|}{\textrm{Zuppa Toscana}}                               \\ 
                                        &                                       \\
    \multicolumn{2}{|l|}{\footnotesize{\textbf{\textrm{Amount per Serving}}}}   \\ \hline
    \textbf{\textrm{Calories (kcal)}}   & \textrm{382.7}                          \\ \hline
    \textbf{\textrm{Fat (g)}}           & \textrm{30.6}                           \\ \hline
    \textbf{\textrm{Carbohydrates (g)}} & \textrm{9.1}                            \\ \hline
    \hspace{2mm} \textrm{Fiber (g)}     & \textrm{2.6}                            \\ \hline
    \textbf{\textrm{Protein (g)}}       & \textrm{17.5}                            \\ \hline
  \end{tabular}

  \freeform{}\hrulefill{}

\end{recipe}

Re-produced from \citetitle{lina_2020} \autocite{lina_2020}

  
\end{document}